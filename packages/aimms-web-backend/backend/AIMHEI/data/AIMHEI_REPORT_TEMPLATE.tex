\documentclass[11pt]{article}
\usepackage{fontspec}
\usepackage{ragged2e}
\usepackage{needspace}
\usepackage{xcolor} % Add color package
\definecolor{green}{RGB}{65, 200, 75} % Define a medium light green color

% Adjust margins
\usepackage[left=0.75in, right=0.75in, top=1in, bottom=2in]{geometry}
\usepackage{fancyhdr}

%\setmainfont{Georgia} % This is now done in report.py

% Define a command for colored variables
\newcommand{\var}[1]{\textcolor{green}{\mbox{#1}}}

% Set text alignment to flush left and prevent hyphenation
\raggedright
\setlength{\parindent}{0pt}  % Remove indentation
\pagestyle{empty}            % Remove page numbers

% Set up the header for the first page
\fancypagestyle{firstpage}{
    \fancyhf{} % Clear all header and footer fields
    \renewcommand{\headrulewidth}{0pt} % Remove the header rule
    % \renewcommand{\footrulewidth}{0pt} % Remove the footer rule if you don't want a line
    % OLD: \rhead{HCP: \var{\{hcp\_name\}} \\ HS: \var{\{human\_supervisor\}} \\ \var{\{date\_today\}}} % Right header
    \fancyfoot[L]{\fontsize{10}{12}\selectfont\textsuperscript{1}\var{\{unacceptable\_performance\_items\}}} % Left-aligned footer
    \setlength{\headheight}{0pt} % Adjust this value as needed
    \setlength{\headsep}{30pt} % Adjust this value as needed
    \setlength{\footskip}{1.25in} % Add this line to control footer position
}

% Set up the header for the second page
\fancypagestyle{plain}{
    \fancyhf{} % clear all header and footer fields
    \renewcommand{\headrulewidth}{0pt}
    \renewcommand{\footrulewidth}{0pt}
}


% Set the space between paragraphs
\setlength{\parskip}{10pt} % Adjust the value to your preferred spacing

\begin{document}

% Apply the 'firstpage' style for the first page
    \thispagestyle{firstpage}

% "AIMHEI REPORT" in bold and 20 point font
    {\fontsize{20pt}{24pt}\selectfont\textbf{AIMHEI REPORT}\par}
    \vspace{-5pt}

% Phrase with bold capital letters
    \textbf{A}rtificially \textbf{I}ntelligent \textbf{M}edical \textbf{H}istory \textbf{E}valuation \textbf{I}nstrument \\Arizona Simulation Technology \& Education Center\\[30pt]

% Body of the document
    % This report relates to an AI-derived evaluation of a medical history interview conducted by healthcare provider (HCP) \var{\{hcp\_name\}} (\var{\{hcp\_year\}}). They interviewed patient ID \var{\{patient\_id\}} on \var{\{date\_today\}} as part of their primary care medicine clerkship. The human supervisor (HS) or faculty member was \var{\{human\_supervisor\}}.

    % The overall score for the medical history interview was \var{\{points\_awarded\}} out of \var{\{points\_total\}} (\var{\{points\_percentage\}}). In summary, \var{\{hcp\_name\}}, \var{\{hcp\_year\}} demonstrated \var{\{adjective\_score\}} proficiency (\var{\{percentile\_score\}} percentile) of the elements of the medical history interview at this stage of their medical training. There were \var{\{unacceptable\_performance\_areas\}}¹. Regardless, areas of improvement are listed on the next page.

    % A full scoring of all areas of the evaluation is available for review.
    % The following summary was generated by an artificial intelligence-enhanced evaluation algorithm. Please feel free to contact the assigned faculty instructor should there be any questions or concerns about the way this was scored by the computer.

    % This evaluation of a medical history was reviewed by \var{\{human\_supervisor\}} and the review is formally attested by the signature below:\\[50pt]

    % \noindent\rule{10cm}{0.4pt}\\
    % Signature\\[50pt]

    % Body of the document
    This report relates to an AI-derived evaluation of a medical history interview conducted by a medical student. The interview took place on \var{\{interview\_date\}}.
    
    The overall score for the medical history interview was \var{\{points\_awarded\}} out of \var{\{points\_total\}} (\var{\{points\_percentage\}}). In summary, this student demonstrated \var{\{adjective\_score\}} proficiency of the elements of the medical history interview at this stage of their medical training. There were \var{\{unacceptable\_performance\_areas\}}¹. Regardless, areas of improvement are listed on the next page.
    
    A full scoring of all areas of the evaluation is available for review. The following summary was generated by an artificial intelligence-enhanced evaluation algorithm. Please note that this program is very strict and no scoring curve has yet been established or applied. This feedback should be considered supplemental and not be used as a replacement for regular societies/human feedback.

    \newpage
    \thispagestyle{plain}

% "AIMES" in bold and 20 point font
    {\fontsize{20pt}{24pt}\selectfont\textbf{AIMHEI SCORING}\par}
    \vspace{10pt}

% Create a table-like structure for the AIMES section
    \begin{table}[h!]  % [h!] tells LaTeX to place the table "here"
        \begin{tabular}{|l|c|c|c|c|}
            \hline
            \textbf{Information Section} & \textbf{Correct} & \textbf{Incorrect} & \textbf{Total} & \textbf{Percentage} \\
            \hline
            \var{\{aimes\_table\}}
            \hline
        \end{tabular}
    \end{table}

    \begin{justify}
        \var{\{aimes\}}
    \end{justify}

\end{document}
